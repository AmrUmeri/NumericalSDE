%% We use the memoir class because it offers a many easy to use features.
\documentclass[10pt,a4paper,titlepage, openany]{memoir} %openright, twosided, slides


\usepackage[OT1]{fontenc}
\usepackage[english]{babel}
\usepackage[utf8]{inputenc}
\usepackage[sc]{mathpazo}
\usepackage[intlimits]{amsmath}
\usepackage{amssymb,amsfonts,mathrsfs}
\usepackage[amsmath,thmmarks]{ntheorem}
\usepackage{graphicx}
\usepackage{pdfpages}
%%Custom
\usepackage{enumitem} 	%Anpassbare Enumerates/Itemizes
%%%\usepackage{appendix}	%self-explaining
%% See the TeXed file for more explanations
\usepackage{subcaption} %Multiple figures, 1 caption

%% [OPT] Multi-rowed cells in tabulars
%\usepackage{multirow}


%% [REC] Intelligent cross reference package. This allows for nice
%% combined references that include the reference and a hint to where
%% to look for it.
\usepackage{varioref}

%% [OPT] Easily changeable quotes with \enquote{Text}
%\usepackage[german=swiss]{csquotes}

%% [REC] Format dates and time depending on locale
\usepackage{datetime}

%% [OPT] Provides a \cancel{} command to stroke through mathematics.
%\usepackage{cancel}

%% [NEED] This allows for additional typesetting tools in mathmode.
%% See its excellent documentation.
\usepackage{mathtools}

%% [ADV] Conditional commands
%\usepackage{ifthen}

%% [OPT] Manual large braces or other delimiters.
%\usepackage{bigdelim, bigstrut}

%% [REC] Alternate vector arrows. Use the command \vv{} to get scaled
%% vector arrows.
\usepackage[h]{esvect}

%% [NEED] Some extensions to tabulars and array environments.
\usepackage{array}

%% [OPT] Postscript support via pstricks graphics package. Very
%% diverse applications.
%\usepackage{pstricks,pst-all}

%% [?] This seems to allow us to define some additional counters.
%\usepackage{etex}

%% [ADV] XY-Pic to typeset some matrix-style graphics
%\usepackage[all]{xy}

%% [OPT] This is needed to generate an index at the end of the
%% document.
%\usepackage{makeidx}

%% [OPT] Fancy package for source code listings.  The template text
%% needs it for some LaTeX snippets; remove/adapt the \lstset when you
%% remove the template content.
\usepackage{listings}
\lstset{language=TeX,basicstyle={\normalfont\ttfamily}}

%% [REC] Fancy character protrusion.  Must be loaded after all fonts.
%\usepackage[activate]{pdfcprot}

%% [REC] Nicer tables.  Read the excellent documentation.
\usepackage{booktabs}

%% Memoir layout setup

%% NOTE: You are strongly advised not to change any of them unless you
%% know what you are doing.  These settings strongly interact in the
%% final look of the document.

% Dependencies
\usepackage{ETHlogo}

% Turn extra space before chapter headings off.
\setlength{\beforechapskip}{0pt}

\nonzeroparskip
\parindent=0pt
\defaultlists

% Chapter style redefinition
\makeatletter

\if@twoside
  \pagestyle{Ruled}
  \copypagestyle{chapter}{Ruled}
\else
  \pagestyle{ruled}
  \copypagestyle{chapter}{ruled}
\fi
\makeoddhead{chapter}{}{}{}
\makeevenhead{chapter}{}{}{}
\makeheadrule{chapter}{\textwidth}{0pt}
\copypagestyle{abstract}{empty}

\makechapterstyle{bianchimod}{%
  \chapterstyle{default}
  \renewcommand*{\chapnamefont}{\normalfont\Large\sffamily}
  \renewcommand*{\chapnumfont}{\normalfont\Large\sffamily}
  \renewcommand*{\printchaptername}{%
    \chapnamefont\centering\@chapapp}
  \renewcommand*{\printchapternum}{\chapnumfont {\thechapter}}
  \renewcommand*{\chaptitlefont}{\normalfont\huge\sffamily}
  \renewcommand*{\printchaptertitle}[1]{%
    \hrule\vskip\onelineskip \centering \chaptitlefont\textbf{\vphantom{gyM}##1}\par}
  \renewcommand*{\afterchaptertitle}{\vskip\onelineskip \hrule\vskip
    \afterchapskip}
  \renewcommand*{\printchapternonum}{%
    \vphantom{\chapnumfont {9}}\afterchapternum}}

% Use the newly defined style
\chapterstyle{bianchimod}

\setsecheadstyle{\Large\bfseries\sffamily}
\setsubsecheadstyle{\large\bfseries\sffamily}
\setsubsubsecheadstyle{\bfseries\sffamily}
\setparaheadstyle{\normalsize\bfseries\sffamily}
\setsubparaheadstyle{\normalsize\itshape\sffamily}
\setsubparaindent{0pt}

% Set captions to a more separated style for clearness
\captionnamefont{\sffamily\bfseries\footnotesize}
\captiontitlefont{\sffamily\footnotesize}
\setlength{\intextsep}{16pt}
\setlength{\belowcaptionskip}{1pt}

% Set section and TOC numbering depth to subsection
\setsecnumdepth{subsection}
\settocdepth{subsection}

%% Titlepage adjustments
\pretitle{\vspace{0pt plus 0.7fill}\begin{center}\HUGE\sffamily\bfseries}
\posttitle{\end{center}\par}
\preauthor{\par\begin{center}\let\and\\\Large\sffamily}
\postauthor{\end{center}}
\predate{\par\begin{center}\Large\sffamily}
\postdate{\end{center}}

\def\@advisors{}
\newcommand{\advisors}[1]{\def\@advisors{#1}}
\def\@department{}
\newcommand{\department}[1]{\def\@department{#1}}
\def\@thesistype{}
\newcommand{\thesistype}[1]{\def\@thesistype{#1}}

\renewcommand{\maketitlehooka}{\indent\ETHlogo[2in]}


\renewcommand{\maketitlehookb}{\vspace{1in}%
  \par\begin{center}\Large\sffamily\@thesistype\end{center}}

\renewcommand{\maketitlehookd}{%
  \vfill\par
  \begin{flushright}
    \sffamily
    \@advisors\par
    \@department, University of Bern
  \end{flushright}
}

\checkandfixthelayout

\setlength{\droptitle}{-48pt}

\makeatother

% This defines how theorems should look. Best leave as is.
\theoremstyle{plain}
\setlength\theorempostskipamount{0pt}

%%% Local Variables:
%%% mode: latex
%%% TeX-master: "thesis"
%%% End:

%% Theorem-like environments

%% This can be changed according to language. You can comment out the ones you
%% don't need.

\numberwithin{equation}{chapter}

%% English variants
\theoremstyle{break}
\newtheorem{theorem}{Theorem}[chapter]
\newtheorem{definition}[theorem]{Definition}
\newtheorem{lemma}[theorem]{Lemma}
\newtheorem{proposition}[theorem]{Proposition}
\newtheorem{algorithm}[theorem]{Algorithm}

%% Custom theorem style
%% Proof environment with a small square as a "qed" symbol
\theoremstyle{nonumberplain}
\theorembodyfont{\normalfont}
\theoremsymbol{\ensuremath{\square}}
\newtheorem{proof}{Proof}
%\newtheorem{beweis}{Beweis}
\theoremstyle{plain}
\newtheorem{example}[theorem]{Example}
\theoremstyle{nonumberplain}
\newtheorem{remark}[theorem]{Remark}
%% Custom commands
%% ===============

%% Special characters for number sets, e.g. real or complex numbers.
\newcommand{\C}{\mathbb{C}}
\newcommand{\K}{\mathbb{K}}
\newcommand{\N}{\mathbb{N}}
\newcommand{\Q}{\mathbb{Q}}
\newcommand{\R}{\mathbb{R}}
\newcommand{\Z}{\mathbb{Z}}
\newcommand{\X}{\mathbb{X}}

%% Fixed/scaling delimiter examples (see mathtools documentation)
\DeclarePairedDelimiter\abs{\lvert}{\rvert}
\DeclarePairedDelimiter\norm{\lVert}{\rVert}

%% Use the alternative epsilon per default and define the old one as \oldepsilon
\let\oldepsilon\epsilon
\renewcommand{\epsilon}{\ensuremath\varepsilon}

%% Also set the alternate phi as default.
\let\oldphi\phi
\renewcommand{\phi}{\ensuremath{\varphi}}

\usepackage[linkcolor=black,colorlinks=true,citecolor=black,filecolor=black]{hyperref}
\usepackage{float}
\begin{document}
\textbf{Titlepage}\\
Welcome everyone to the presentation of my bachelor thesis.\\
My thesis is about SDE and how to construct numerical methods for the approximative solution.

\textbf{Overview}\\
I would like to give an overview of my thesis:\\
My thesis includes 3 parts:\\
\begin{enumerate}
\item In the first part I discuss the theory of stochastic integration.\\
This is a theory of integration w.r.t stochastic processes. This means the integrator is a stochastic process and also the function
which we want to integrate is allowed to be a stochastic process. I will focus on the theory of It\^o.
\item Then I use these stochastic integrals in order to define so-called SDE's.\\
I will tell more in a moment.
\item In the final part I construct methods in order to solve these equations numerically.\\
These methods are based on stochastic integration theory.\\
Then I also studied the convergence properties of these methods.
\end{enumerate}

\textbf{Introduction}\\
A SDE is a differential equation which includes a random term.\\ 
Often it is given in symbolical form\\
Usually if you have an ODE, you are dividing over dt. But here you are not allowed to divide over dt, because the random term may be non-differentable. \\
The mathematical interpretation of the equation is given in integral form. It is a sum of a Riemann-integral and a so-called stochastic integral. The second integral is the stochastic integral. And the randomness in the equation is included through this integral.\\
This means that also the solution of such equations must include some randomness.

\textbf{Introduction}\\
The problems which arise are the following:
\begin{enumerate}
\item How to define such equations and especially how to define the stochastic integral.
\item What are the solutions and how to get them analytically or numerically.
\item What are the conditions for the existence or uniqueness of the solution.
\end{enumerate}

\newpage

\textbf{Introduction}\\
If the solution exists, it is a stochastic process.\\
In the graphic you can see one possible realization of a stochastic process. I will define stochastic processes later.


\textbf{Main goal}\\
However often we cannot solve these equations analytically and then we would like to have numerical approximations of the solution.\\
In my thesis I discussed so-called strong methods. These are methods which give an approximation of the sample paths of the solution.\\
And then we would also like to compare different methods w.r.t their convergence speed. Some methods are faster than others. Here we need the concept of convergence order.\\
In the graphic you can see possible approximations. For example in Blue you can see the Euler-method. This is a generalization of the Euler-method for ordinary differential equations. In red you can see the Milstein-method.



\textbf{Terminology}\\
Now I would like to introduce some concepts which we need to understand stochastic integration.
\begin{enumerate}
\item We have given a probability space.\\
We define a stochastic process as a collection of random variables indexed by some set. Here we take the inverval [0,T] as the index set and you can understand this as some time-interval.\\
I will also use the following notation.\\
You can understand stochastic processes also like this:\\
If you fix the time t, then you get a random variable.
\end{enumerate}

\textbf{Terminology}\\
Also if you fix an omega in your event set, you can define the following mapping.\\
We call these the realizations of a process.\\
In the graphic you can see some possible realizations of a continuous stochastic process.


\textbf{Terminology}\\
Now I would like to present the Wiener process. The Wiener process is a stochastic process. It is important to us because in stochastic integration theory you are integrating w.r.t the Wiener process.\\
The Wiener process starts at 0. It has normal-distributed increments and the increments are independent if you have disjoint intervals.\\
At each time the Wiener process is a normal-distributed random variable.

\textbf{Terminology}\\
In the graphic you can see possible realizations of the Wiener process.\\
Here I had to give a discretization of the Wiener process.

\newpage

\textbf{The first Stochastic integral}
Now we will discuss a first example of a stochastic integral.\\
Here we want to integrate the Wiener process w.r.t itself, and 2 is just a factor.\\
Similar to Riemann-integration we start with giving a partition of [0,T].\\
Then we give 2 possible approximations of the stochastic integral.
\begin{enumerate}
\item In the first approximation you take the Wiener process at the beginning of each interval.
\item In the second approximation you take the Wiener process at the end of each interval.
\end{enumerate}
Then you take the sum.\\
The question is then, what happens if n goes to infinity, this means what happens if your partition get finer and finer.\\
And what kind of convergence do we have.\\

\textbf{The first Stochastic integral}
It is possible to show that a limit is taken in \(L_2\) (EXPLAIN) if n goes to infinity. But the limits are different.\\
This is different to Riemann-Integration where you had that upper and lower sum approximation converge to the same limit.\\
In stochastic integration theory it depends how you give the approximations.


\textbf{Questions}
The first definition is also known as It\^o stochastic integration. Recall that for a given partition we take the process at the left on each interval. In my thesis I focussed on this definition.\\

Then the question is: for which stochastic processes can we define the It\^o stochastic integral?


\textbf{Terminology}\\
To answer this question I need to introduce some concepts:
\begin{enumerate}
\item The first one is filtration.\\
Each random variable generates a sigma-algebra.(EXPLAIN).\\
This means a stochastic process is generating a sequence of sigma-algebras, which is also called filtration.\\
Here we are interested in the filtration generated by the Wiener process.
\item Then we will another process \(X_t\) to be adapted if it is measurable w.r.t filtration generated by the Wiener process.
\end{enumerate}



\textbf{Terminology}\\
It is possible to see stochastic processes as elements in some space then it is possible to define a norm on this space. Here we define the \(L_2\)-norm for stochastic processes like this.\\
Then I would like to define a class of stochastic processes C, with the following properties:
\begin{enumerate}
\item The process is measurable, this is already by definition
\item The \(L_2\)-norm of the stochastic process is finite
\item The stochastic process is adapted to the filtration generated by the Wiener process
\end{enumerate}


\textbf{Stochastic integration}\\
It is possible to show the following result:\\
For each stochastic process in our class of functions which we defined previously, the It\^o-stochastic integral exists as an element of \(L_2\)-Omega. This means it is a random variable and has finite variance.\\
Here the difficulty is to show that the stochastic integral is measurable.\\

The construction is done the following way:\\
First we define the stochastic integral for so-called random step functions, there are stochastic processes which have step functions as sample paths.\\
Then we can use these to approximate the stochastic integral for any stochastic process in the class C.\\
Here we need to use that the class of random step functions is dense in C.

\textbf{Stochastic integration}\\
Now I would like to talk about an important result which holds only for It\^o stochastic integral.\\
The It\^o-isometry tells you that the \(L_2\)-norm of your stochastic process is the same as the \(L_2\)-norm of the stochastic integral.\\
This result is needed for the construction of It\^o stochastic integrals.


\textbf{It\^o-lemma}\\
The It\^o-lemma is the main result of stochastic integration theory.\\
If you have a stochastic differential equation defined like this and you have a smooth map u.\\
Then the question is: What kind of stochastic differential equation does the mapping \(u(X_t)\) solve?\\
The lemma gives you an answer to this the following way:

\textbf{It\^o-lemma}\\
\(u(X_t)\) will look like this basically. This is also how you get the solution of stochastic differential equations. Sometimes you can choose the function u such that you get the solution \(X_t\) of the original stochastic differential equation. 
we will see examples later.

\newpage
\textbf{Stochastic Differential Equation}\\
Now let us consider again stochastic differential equations. It is possible to give conditions on the functions a and b such that the solution exists and is unique.\\
Now I would like to give some examples of stochastic differential equations.

\textbf{Geometric brownian motion}\\
If we set \(a(x)\) equal to mu x and \(b(x)\) equal to sigma x, mu, sigma constants. Then we get the following SDE.\\
It is possible to get the solution using the It\^o-lemma.\\
The solution is also called geometric brownian motion. It is a continuous stochastic process such that at each time the process has a lognormal distribution.

\textbf{Ornstein-Uhlenbeck process}\\
The Ornstein-Uhlenbeck process is another example.\\
Here \(a(x)\) is equal to -beta x and \(b(x)\) is equal to sigma , constant.\\
Also here we can get the solution using the It\^o-lemma.

\newpage

\textbf{Stochastic Taylor expansions}\\
Now I would like to discuss an important application of the It\^o-lemma, so-called stochastic taylor expansions.\\
We consider again a SDE then we apply the It\^o-lemma on \(a(x)\) and \(b(x)\).\\

\textbf{Stochastic Taylor expansions}\\
This gives us the following expansions for \(a(x)\) and \(b(x)\).

\textbf{Stochastic Taylor expansions}\\
Then we plug these expansions into the original SDE. This gives us the following equation.\\
Actually it gives you a very long expression but I summarized some terms into the variable R, this is the remainder term.\\
This is also called the first stochastic taylor expansion.

\textbf{Stochastic Taylor expansions}\\
If you apply the It\^o-lemma a third time on one of the terms in R, then you get the following expansion, here we just recieved an additional term and R is again the remainder term\\
This is also called the second stochastic taylor expansion.

\textbf{Stochastic Taylor expansions}\\
Now the idea of numerical algorithms for SDE's is just to give an expansion of the equation using the It\^o-lemma and then to remove terms which have a high order w.r.t to the \(L_2\)-norm, in our case we always will remove the remainder terms.


\textbf{Euler-Method for SDE's}\\
For example the Euler-method is based on the first stochastic taylor expansion. We give the expansion for n points \(t_0\cdots t_n\) and remove the remainder term.\\
The algorithm is defined recursively.

\textbf{Euler-Method for SDE's}\\
The second stochastic taylor expansions leads to the Milstein-method, it is defined in the same way as the Euler-method.

\newpage


\textbf{Illustration}\\
Now I would like to give an illustration of both methods.\\
On the left you can see the Euler-Approximation in blue. The true solution is in black. And the coloread area is the difference between the true solution and the approximated solution.\\
On the right you can see the same but here the approximation has beed done using the Milstein-method.\\

Here we can see already that the Milstein-method is better in some sense than the Euler-Method.

\textbf{Numerical Analysis}\\
Now I would like to give a criterion in order to measure the approximation error.\\
You can do this in different ways.\\
A possibility is to take the difference between the approximation and the true solution at the final time T in the \(L_2\)-norm.

\textbf{Numerical Analysis}\\
Now we would like to know how fast our methods are converging. It is possible to give an error bound the following way\\
Then the constant gamma determines the convergence speed of our methods. Thats why gamma is called the order of convergence.\\
There are 2 ways to estimate gamma\\
You can give an analytical error bound\\
or you can estimate gamma using simulation.\\


\textbf{Numerical Analysis}\\
It is possible to show that the Euler-method has order of convergence 0.5\\
and the Milstein-method has order of convergence 1.0\\
This means that the Milstein-scheme converges faster.























\end{document}
